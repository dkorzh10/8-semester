\documentclass[12pt]{article}
\usepackage[utf8]{inputenc}
\usepackage[russian]{babel}
\usepackage[T2A]{fontenc}
\usepackage[top=2cm,right=2cm,left=2cm,bottom=2cm]{geometry}

\begin{document}
\title{Теоретический минимум по оптимизации}
\author{}
\date{}

\maketitle
\thispagestyle{empty}
\vspace{-2cm}
\begin{enumerate}
\item Общая постановка задачи оптимизации
\item Что такое выпуклое множество? Что такое аффинное множество? Что такое конус, выпуклый конус?
\item Что такое выпуклая, неотрицательная, аффинная комбинации точек?
\item Что такое относительная внутренность множества?
\item Какие два множества называются отделимыми?
\item Что такое опорная гиперплоскость к множеству?
\item Что такое проекция точки на множество?
\item Что такое сопряжённое множество и сопряжённый конус?
\item Что такое градиент, матрица Якоби и гессиан функции?
\item Что такое выпуклая и сильно выпуклая функция?
\item Что такое надграфик и множество подуровней?
\item Что такое субдифференциал и условный субдифференциал? Что такое нормальный конус?
\item Функция Лагранжа
\item Условия ККТ и Слейтера
\item Задача линейного программирования и двойственная к ней
\item Задача квадратичного программирования и двойственная к ней

\end{enumerate}
\end{document}